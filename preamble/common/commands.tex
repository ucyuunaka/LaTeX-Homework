% =============================================================================
%                    通用命令与环境定义 - Common Commands
% =============================================================================

% ========== 定理环境设置 ==========
\theoremstyle{definition} 
\newtheorem{definition}{定义}[section] 
\newtheorem{theorem}{定理}[section] 
\newtheorem{lemma}{引理}[section] 
\newtheorem{corollary}{推论}[section] 
\newtheorem{example}{例}[section] 
\newtheorem{remark}{注}[section] 

% ========== 链接样式设置 ==========
\hypersetup{colorlinks=true,linkcolor=black,citecolor=green,urlcolor=blue} 

% ========== 代码样式配置 ==========
\definecolor{codebg}{rgb}{0.95,0.95,0.95} 
\definecolor{commentcolor}{rgb}{0.0,0.5,0.0} 
\definecolor{keywordcolor}{rgb}{0.0,0.0,0.8} 
\definecolor{stringcolor}{rgb}{0.8,0.0,0.0} 

\setminted{
    bgcolor=codebg,
    linenos,
    breaklines,
    breakanywhere,
    fontsize=\small,
    frame=single,
    framesep=2mm
} 

\lstset{
    backgroundcolor=\color{codebg},
    basicstyle=\ttfamily\small,
    breakatwhitespace=false,
    breaklines=true,
    captionpos=b,
    commentstyle=\color{commentcolor}\itshape,
    escapeinside={\%*}{*)},
    extendedchars=true,
    frame=single,
    keepspaces=true,
    keywordstyle=\color{keywordcolor}\bfseries,
    language=Python,
    numbers=left, 
    numbersep=5pt, 
    numberstyle=\tiny\color{gray}, 
    rulecolor=\color{black}, 
    showspaces=false, 
    showstringspaces=false, 
    showtabs=false, 
    stepnumber=1, 
    stringstyle=\color{stringcolor}, 
    tabsize=4, 
    title=\lstname, 
    postbreak=\mbox{\textcolor{red}{$\hookrightarrow$}\space}, 
    xleftmargin=2em, 
    framexleftmargin=1.5em 
}

\lstdefinelanguage{JavaScript}{
    keywords={break, case, catch, continue, debugger, default, delete, do, else, false, finally, for, function, if, in, instanceof, new, null, return, switch, this, throw, true, try, typeof, var, void, while, with, const, let, class, export, import, async, await}, 
    morecomment=[l]{//}, 
    morecomment=[s]{/*}{*/}, 
    morestring=[b]', 
    morestring=[b]", 
    morestring=[b]`, 
    ndkeywords={class, export, boolean, throw, implements, import, this}, 
    keywordstyle=\color{keywordcolor}\bfseries, 
    ndkeywordstyle=\color{purple}\bfseries, 
    identifierstyle=\color{black}, 
    commentstyle=\color{commentcolor}\ttfamily, 
    stringstyle=\color{stringcolor}\ttfamily, 
    sensitive=true 
}

\SetAlgorithmName{算法}{算法}{算法列表} 
\SetKwProg{Fn}{函数}{:}{结束} 
\SetKwFunction{FMain}{主函数} 

\captionsetup{
    font={small,bf},
    labelfont=bf,
    textfont=rm,
    justification=centering
} 

\setlist[itemize]{leftmargin=2em,itemsep=0.5ex} 
\setlist[enumerate]{leftmargin=2em,itemsep=0.5ex} 

% ========== 自定义命令 ==========
\newcommand{\highlight}[1]{\colorbox{yellow}{#1}} 

\newcommand{\notebox}[1]{% 
    \begin{center} 
    \fcolorbox{red}{yellow!20}{% 
        \parbox{0.9\textwidth}{% 
            \textbf{注意:}#1 
        } 
    } 
    \end{center} 
}

\newcommand{\insertfig}[4][0.8]{% 
    \begin{figure}[H] 
        \centering 
        \includegraphics[width=#1\textwidth]{#2} 
        \caption{#3} 
        \label{#4} 
    \end{figure} 
}

\newcommand{\inputcode}[3][python]{% 
    \begin{listing}[H] 
        \inputminted{#1}{#2} 
        \caption{#3} 
    \end{listing} 
}

\makeatletter 
\newcommand\dlmu[2][4cm]{% 
  \sbox0{#2}% 
  \ifdim\wd0>#1\relax 
    \hskip1pt\underline{\hb@xt@ #1{\hss\resizebox*{#1}{!}{\mbox{#2}}\hss}}\hskip3pt% 
  \else 
    \hskip1pt\underline{\hb@xt@ #1{\hss#2\hss}}\hskip3pt% 
  \fi 
}
\makeatother