% =============================================================================
% 双栏浮动体专用设置文件
% Double Column Float Specific Configuration
% =============================================================================
% 文件用途:双栏布局中浮动体(图表)的专用配置
% 使用说明:
% 1. 本文件专门配置双栏布局中的浮动体行为
% 2. 在主文档的 preamble 中使用 % =============================================================================
% 双栏浮动体专用设置文件
% Double Column Float Specific Configuration
% =============================================================================
% 文件用途:双栏布局中浮动体(图表)的专用配置
% 创建日期:2025-08-03
% 版本:v1.0
% 
% 使用说明:
% 1. 本文件专门配置双栏布局中的浮动体行为
% 2. 在主文档的 preamble 中使用 % =============================================================================
% 双栏浮动体专用设置文件
% Double Column Float Specific Configuration
% =============================================================================
% 文件用途:双栏布局中浮动体(图表)的专用配置
% 创建日期:2025-08-03
% 版本:v1.0
% 
% 使用说明:
% 1. 本文件专门配置双栏布局中的浮动体行为
% 2. 在主文档的 preamble 中使用 % =============================================================================
% 双栏浮动体专用设置文件
% Double Column Float Specific Configuration
% =============================================================================
% 文件用途:双栏布局中浮动体(图表)的专用配置
% 创建日期:2025-08-03
% 版本:v1.0
% 
% 使用说明:
% 1. 本文件专门配置双栏布局中的浮动体行为
% 2. 在主文档的 preamble 中使用 \input{preamble/layout_double/floats.tex} 引入
% 3. 建议在引入 layout_double/settings.tex 之后引入本文件
% 4. 本文件解决双栏布局中图表放置的特殊需求
% 
% 配置内容包括:
% - 跨栏浮动体设置(figure*、table* 环境)
% - 单栏内浮动体位置控制
% - 浮动体与文本的间距调整
% - 图表标题在双栏中的格式设置
% - 浮动体排队和放置算法优化
% - 特殊浮动体环境定义
% 
% 注意事项:
% - 此配置仅适用于双栏布局
% - 跨栏浮动体会影响页面布局,需谨慎使用
% - 某些设置可能与栏平衡功能产生冲突
% - 建议配合 balance.tex 一起调试
% =============================================================================

% 此文件将在后续迁移过程中填充具体的双栏浮动体配置内容 引入
% 3. 建议在引入 layout_double/settings.tex 之后引入本文件
% 4. 本文件解决双栏布局中图表放置的特殊需求
% 
% 配置内容包括:
% - 跨栏浮动体设置(figure*、table* 环境)
% - 单栏内浮动体位置控制
% - 浮动体与文本的间距调整
% - 图表标题在双栏中的格式设置
% - 浮动体排队和放置算法优化
% - 特殊浮动体环境定义
% 
% 注意事项:
% - 此配置仅适用于双栏布局
% - 跨栏浮动体会影响页面布局,需谨慎使用
% - 某些设置可能与栏平衡功能产生冲突
% - 建议配合 balance.tex 一起调试
% =============================================================================

% 此文件将在后续迁移过程中填充具体的双栏浮动体配置内容 引入
% 3. 建议在引入 layout_double/settings.tex 之后引入本文件
% 4. 本文件解决双栏布局中图表放置的特殊需求
% 
% 配置内容包括:
% - 跨栏浮动体设置(figure*、table* 环境)
% - 单栏内浮动体位置控制
% - 浮动体与文本的间距调整
% - 图表标题在双栏中的格式设置
% - 浮动体排队和放置算法优化
% - 特殊浮动体环境定义
% 
% 注意事项:
% - 此配置仅适用于双栏布局
% - 跨栏浮动体会影响页面布局,需谨慎使用
% - 某些设置可能与栏平衡功能产生冲突
% - 建议配合 balance.tex 一起调试
% =============================================================================

% 此文件将在后续迁移过程中填充具体的双栏浮动体配置内容 引入
% 3. 建议在引入 layout_double/settings.tex 之后引入本文件
% 4. 本文件解决双栏布局中图表放置的特殊需求
% 
% 配置内容包括:
% - 跨栏浮动体设置(figure*、table* 环境)
% - 单栏内浮动体位置控制
% - 浮动体与文本的间距调整
% - 图表标题在双栏中的格式设置
% - 浮动体排队和放置算法优化
% - 特殊浮动体环境定义
% 
% 注意事项:
% - 此配置仅适用于双栏布局
% - 跨栏浮动体会影响页面布局,需谨慎使用
% - 某些设置可能与栏平衡功能产生冲突
% - 建议配合 balance.tex 一起调试
% =============================================================================

% ========== 双栏浮动体参数设置 ==========

% --- 浮动体放置策略 ---
% 设置浮动体的默认放置参数
\renewcommand{\topfraction}{0.9}        % 页面顶部浮动体最大占比
\renewcommand{\bottomfraction}{0.8}     % 页面底部浮动体最大占比
\renewcommand{\textfraction}{0.1}       % 页面文本最小占比
\renewcommand{\floatpagefraction}{0.8}  % 浮动页浮动体最小占比
\setcounter{topnumber}{3}               % 页面顶部最多浮动体数量
\setcounter{bottomnumber}{2}            % 页面底部最多浮动体数量
\setcounter{totalnumber}{4}             % 每页最多浮动体总数

% --- 双栏内浮动体设置 ---
% 图片默认浮动策略:就近放置
\makeatletter
\renewcommand{\fps@figure}{!htb}        % 图片浮动顺序:强制here, top, bottom
\renewcommand{\fps@table}{!htb}         % 表格浮动顺序:强制here, top, bottom
\makeatother

% --- 跨栏浮动体设置 ---
% 跨栏图片和表格的默认设置
% 注意:在某些情况下这些命令可能未定义,需要条件检查
\makeatletter
\@ifundefined{fps@figure*}{}{%
    \renewcommand{\fps@figure*}{!tb}        % 跨栏图片:top, bottom(不允许here)
}
\@ifundefined{fps@table*}{}{%
    \renewcommand{\fps@table*}{!tb}         % 跨栏表格:top, bottom(不允许here)
}
\makeatother

% --- 浮动体间距调整 ---
% 浮动体与文本之间的间距(双栏布局需要更紧凑)
\setlength{\floatsep}{8pt plus 2pt minus 2pt}        % 同一页面浮动体间距
\setlength{\textfloatsep}{12pt plus 2pt minus 4pt}   % 浮动体与文本间距
\setlength{\intextsep}{8pt plus 2pt minus 2pt}       % 文中浮动体间距
\setlength{\dblfloatsep}{8pt plus 2pt minus 2pt}     % 跨栏浮动体间距
\setlength{\dbltextfloatsep}{12pt plus 2pt minus 4pt} % 跨栏浮动体与文本间距

% ========== 浮动体环境优化 ==========

% --- 双栏内小图片环境 ---
\newcommand{\insertfigsmall}[4][0.7]{%
    \begin{figure}[!htb]
        \centering
        \includegraphics[width=#1\columnwidth]{#2}
        \caption{#3}
        \label{#4}
    \end{figure}
}

% --- 双栏内紧凑表格环境 ---
\newcommand{\inserttablecompact}[3]{%
    \begin{table}[!htb]
        \centering
        \footnotesize  % 使用小字体
        #1
        \caption{#2}
        \label{#3}
    \end{table}
}

% --- 跨栏大表格环境 ---
\newcommand{\inserttablewide}[3]{%
    \begin{table*}[!tb]
        \centering
        \small  % 使用小字体
        #1
        \caption{#2}
        \label{#3}
    \end{table*}
}

% --- 并排子图环境(双栏内) ---
\newcommand{\insertsubfigs}[9][0.45]{%
    \begin{figure}[!htb]
        \centering
        \begin{subfigure}{#1\columnwidth}
            \centering
            \includegraphics[width=\textwidth]{#2}
            \caption{#3}
            \label{#4}
        \end{subfigure}
        \hfill
        \begin{subfigure}{#1\columnwidth}
            \centering
            \includegraphics[width=\textwidth]{#5}
            \caption{#6}
            \label{#7}
        \end{subfigure}
        \caption{#8}
        \label{#9}
    \end{figure}
}

% ========== 浮动体排版优化 ==========

% --- 图表标题格式 ---
% 双栏布局中的标题格式调整
\captionsetup[figure]{
    font=footnotesize,
    labelfont=bf,
    textfont=rm,
    justification=centering,
    singlelinecheck=false,
    skip=6pt
}

\captionsetup[table]{
    font=footnotesize,
    labelfont=bf,
    textfont=rm,
    justification=centering,
    singlelinecheck=false,
    skip=6pt
}

% --- 跨栏图表标题格式 ---
\DeclareCaptionStyle{widefigure}{
    font=small,
    labelfont=bf,
    textfont=rm,
    justification=centering,
    singlelinecheck=false,
    skip=8pt
}

\captionsetup[figure*]{style=widefigure}
\captionsetup[table*]{style=widefigure}

% ========== 特殊浮动体处理 ==========

% --- 侧边注释框(仅限单栏内) ---
\newcommand{\sidenotebox}[1]{%
    \begin{figure}[!htb]
        \centering
        \fcolorbox{blue}{blue!10}{%
            \parbox{0.9\columnwidth}{%
                \footnotesize
                \textbf{提示:}#1
            }
        }
    \end{figure}
}

% --- 算法环境适配双栏 ---
\floatstyle{ruled}
\newfloat{algorithm}{!htb}{loa}
\floatname{algorithm}{算法}

% 双栏算法环境
\newcommand{\insertalgorithm}[3]{%
    \begin{algorithm}[!htb]
        \centering
        \footnotesize
        #1
        \caption{#2}
        \label{#3}
    \end{algorithm}
}

% 跨栏算法环境
\newfloat{algorithm*}{!tb}{loa}
\floatname{algorithm*}{算法}

\newcommand{\insertalgorithmwide}[3]{%
    \begin{algorithm*}[!tb]
        \centering
        \small
        #1
        \caption{#2}
        \label{#3}
    \end{algorithm*}
}

% ========== 浮动体调试命令 ==========
% 用于调试浮动体位置的命令
\newcommand{\showfloatstatus}{%
    \typeout{当前浮动体状态:}%
    \typeout{  topnumber: \the\c@topnumber}%
    \typeout{  bottomnumber: \the\c@bottomnumber}%
    \typeout{  totalnumber: \the\c@totalnumber}%
    \typeout{  topfraction: \topfraction}%
    \typeout{  bottomfraction: \bottomfraction}%
    \typeout{  textfraction: \textfraction}%
}