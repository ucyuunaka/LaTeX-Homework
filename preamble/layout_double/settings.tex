% =============================================================================
%                    双栏布局配置 - Double Column Layout
% =============================================================================

% ========== 文档类设置 ==========
\documentclass[9pt,twocolumn]{ctexart}

% ========== 引入通用配置 ==========
% =============================================================================
%                    通用包导入配置 - Common Packages
% =============================================================================

% --- 页面与版式 ---
\RequirePackage{geometry} 
\RequirePackage{fancyhdr} 
\RequirePackage{microtype} 
\RequirePackage{setspace}      % 行距控制 

% --- 图形与表格 ---
\RequirePackage{graphicx} 
\RequirePackage{booktabs} 
\RequirePackage[table]{xcolor} 
\RequirePackage{tabularx} 
\RequirePackage{multirow} 
\RequirePackage{makecell} 
\RequirePackage{subcaption} 
\RequirePackage{float} 
\RequirePackage{longtable}     % 跨页表格 
\RequirePackage{array}         % 增强表格功能 

% --- 数学与符号 ---
\RequirePackage{amsmath} 
\RequirePackage{amsthm}        % 定理环境 
\RequirePackage{amssymb}       % 数学符号 
\RequirePackage{mathtools}     % 数学工具增强 
\RequirePackage{siunitx} 
\RequirePackage{gensymb} 

% --- 算法与流程图 ---
\RequirePackage[ruled,vlined,linesnumbered]{algorithm2e} 
\RequirePackage{tikz}          % 绘图包 
\usetikzlibrary{shapes,arrows,positioning,calc} 

% --- 数据可视化 ---
\RequirePackage{pgfplots} 
\pgfplotsset{compat=1.18} 

% --- 代码与文本处理 ---
\RequirePackage[cache=false]{minted} 
\RequirePackage{listings} 
\RequirePackage{seqsplit} 
\RequirePackage[hyphens]{url} 
\RequirePackage{xurl} 

% --- 链接与引用 ---
\RequirePackage[colorlinks,bookmarksopen,bookmarksnumbered,breaklinks=true]{hyperref} 
\RequirePackage{cleveref} 
\RequirePackage[
    backend=biber,
    style=gb7714-2015,
    citestyle=numeric-comp,
    sorting=ynt,
    hyperref=true, 
    backref=false
]{biblatex} 

% --- 特殊包 ---
\RequirePackage{enumitem}      % 列表自定义 
\RequirePackage{caption}       % 标题自定义 
\RequirePackage{footnote}      % 脚注增强 
\RequirePackage{appendix}      % 附录管理 
\RequirePackage{subfiles}      % 子文件支持 
\RequirePackage{calc}          % 计算支持
% =============================================================================
%                    通用命令与环境定义 - Common Commands
% =============================================================================

% ========== 定理环境设置 ==========
\theoremstyle{definition} 
\newtheorem{definition}{定义}[section] 
\newtheorem{theorem}{定理}[section] 
\newtheorem{lemma}{引理}[section] 
\newtheorem{corollary}{推论}[section] 
\newtheorem{example}{例}[section] 
\newtheorem{remark}{注}[section] 

% ========== 链接样式设置 ==========
\hypersetup{colorlinks=true,linkcolor=black,citecolor=green,urlcolor=blue} 

% ========== 代码样式配置 ==========
\definecolor{codebg}{rgb}{0.95,0.95,0.95} 
\definecolor{commentcolor}{rgb}{0.0,0.5,0.0} 
\definecolor{keywordcolor}{rgb}{0.0,0.0,0.8} 
\definecolor{stringcolor}{rgb}{0.8,0.0,0.0} 

\setminted{
    bgcolor=codebg,
    linenos,
    breaklines,
    breakanywhere,
    fontsize=\small,
    frame=single,
    framesep=2mm
} 

\lstset{
    backgroundcolor=\color{codebg},
    basicstyle=\ttfamily\small,
    breakatwhitespace=false,
    breaklines=true,
    captionpos=b,
    commentstyle=\color{commentcolor}\itshape,
    escapeinside={\%*}{*)},
    extendedchars=true,
    frame=single,
    keepspaces=true,
    keywordstyle=\color{keywordcolor}\bfseries,
    language=Python,
    numbers=left, 
    numbersep=5pt, 
    numberstyle=\tiny\color{gray}, 
    rulecolor=\color{black}, 
    showspaces=false, 
    showstringspaces=false, 
    showtabs=false, 
    stepnumber=1, 
    stringstyle=\color{stringcolor}, 
    tabsize=4, 
    title=\lstname, 
    postbreak=\mbox{\textcolor{red}{$\hookrightarrow$}\space}, 
    xleftmargin=2em, 
    framexleftmargin=1.5em 
}

\lstdefinelanguage{JavaScript}{
    keywords={break, case, catch, continue, debugger, default, delete, do, else, false, finally, for, function, if, in, instanceof, new, null, return, switch, this, throw, true, try, typeof, var, void, while, with, const, let, class, export, import, async, await}, 
    morecomment=[l]{//}, 
    morecomment=[s]{/*}{*/}, 
    morestring=[b]', 
    morestring=[b]", 
    morestring=[b]`, 
    ndkeywords={class, export, boolean, throw, implements, import, this}, 
    keywordstyle=\color{keywordcolor}\bfseries, 
    ndkeywordstyle=\color{purple}\bfseries, 
    identifierstyle=\color{black}, 
    commentstyle=\color{commentcolor}\ttfamily, 
    stringstyle=\color{stringcolor}\ttfamily, 
    sensitive=true 
}

\SetAlgorithmName{算法}{算法}{算法列表} 
\SetKwProg{Fn}{函数}{:}{结束} 
\SetKwFunction{FMain}{主函数} 

\captionsetup{
    font={small,bf},
    labelfont=bf,
    textfont=rm,
    justification=centering
} 

\setlist[itemize]{leftmargin=2em,itemsep=0.5ex} 
\setlist[enumerate]{leftmargin=2em,itemsep=0.5ex} 

% ========== 自定义命令 ==========
\newcommand{\highlight}[1]{\colorbox{yellow}{#1}} 

\newcommand{\notebox}[1]{%
    \begin{center}
    \fcolorbox{red}{yellow!20}{%
        \parbox{\noteboxwidth}{%
            \textbf{注意:}#1
        }
    }
    \end{center}
}

\newcommand{\insertfig}[4][0.8]{% 
    \begin{figure}[H] 
        \centering 
        \includegraphics[width=#1\textwidth]{#2} 
        \caption{#3} 
        \label{#4} 
    \end{figure} 
}

\newcommand{\inputcode}[3][python]{% 
    \begin{listing}[H] 
        \inputminted{#1}{#2} 
        \caption{#3} 
    \end{listing} 
}

\makeatletter 
\newcommand\dlmu[2][4cm]{% 
  \sbox0{#2}% 
  \ifdim\wd0>#1\relax 
    \hskip1pt\underline{\hb@xt@ #1{\hss\resizebox*{#1}{!}{\mbox{#2}}\hss}}\hskip3pt% 
  \else 
    \hskip1pt\underline{\hb@xt@ #1{\hss#2\hss}}\hskip3pt% 
  \fi 
}
\makeatother

% ========== 双栏布局特定设置 ==========

% --- 页面格式设置 ---
\setlength{\parindent}{1.5em}  % 双栏布局缩进稍小
\geometry{
    left=1.5cm,
    right=1.5cm,
    top=2cm,
    bottom=2cm,
    headsep=1cm,
    columnsep=0.8cm  % 栏间距
} 
\pagestyle{fancy} 
\setlength{\headheight}{28pt}  % 双栏布局头部稍小
\renewcommand{\headrulewidth}{0.4pt} 
\renewcommand{\footrulewidth}{0pt} 
\setstretch{1.2}  % 双栏布局行距紧凑

% --- 双栏图片尺寸优化 ---
% 重新定义 insertfig 命令以适应双栏布局
\renewcommand{\insertfig}[4][0.9]{%
    \begin{figure}[H] 
        \centering 
        \includegraphics[width=#1\columnwidth]{#2} 
        \caption{#3} 
        \label{#4} 
    \end{figure} 
}

% 定义跨栏图片命令
\newcommand{\insertfigwide}[4][0.8]{%
    \begin{figure*}[!t] 
        \centering 
        \includegraphics[width=#1\textwidth]{#2} 
        \caption{#3} 
        \label{#4} 
    \end{figure*} 
}

% --- 双栏代码块优化 ---
% 调整代码块样式以适应双栏
\setminted{
    bgcolor=codebg,
    linenos=false,  % 双栏时关闭行号节省空间
    breaklines,
    breakanywhere,
    fontsize=\footnotesize,  % 字体更小
    frame=single,
    framesep=1mm  % 边距更小
}

\lstset{
    backgroundcolor=\color{codebg},
    basicstyle=\ttfamily\footnotesize,  % 字体更小
    breakatwhitespace=false,
    breaklines=true,
    captionpos=b,
    commentstyle=\color{commentcolor}\itshape,
    escapeinside={\%*}{*)},
    extendedchars=true,
    frame=single,
    keepspaces=true,
    keywordstyle=\color{keywordcolor}\bfseries,
    language=Python,
    numbers=none,  % 双栏时关闭行号
    rulecolor=\color{black}, 
    showspaces=false, 
    showstringspaces=false, 
    showtabs=false, 
    stepnumber=1, 
    stringstyle=\color{stringcolor}, 
    tabsize=4, 
    title=\lstname, 
    postbreak=\mbox{\textcolor{red}{$\hookrightarrow$}\space}, 
    xleftmargin=1em,  % 左边距更小
    framexleftmargin=0.5em 
}

% --- 双栏表格优化 ---
% 调整标题样式以适应双栏
\captionsetup{
    font={footnotesize,bf},  % 字体更小
    labelfont=bf,
    textfont=rm,
    justification=centering
}

% --- 双栏列表优化 ---
\setlist[itemize]{leftmargin=1.5em,itemsep=0.3ex}  % 间距更紧凑
\setlist[enumerate]{leftmargin=1.5em,itemsep=0.3ex}

% --- 引入双栏专用模块 ---
% =============================================================================
% 双栏浮动体专用设置文件
% Double Column Float Specific Configuration
% =============================================================================
% 文件用途:双栏布局中浮动体(图表)的专用配置
% 创建日期:2025-08-03
% 版本:v1.0
% 
% 使用说明:
% 1. 本文件专门配置双栏布局中的浮动体行为
% 2. 在主文档的 preamble 中使用 % =============================================================================
% 双栏浮动体专用设置文件
% Double Column Float Specific Configuration
% =============================================================================
% 文件用途:双栏布局中浮动体(图表)的专用配置
% 创建日期:2025-08-03
% 版本:v1.0
% 
% 使用说明:
% 1. 本文件专门配置双栏布局中的浮动体行为
% 2. 在主文档的 preamble 中使用 % =============================================================================
% 双栏浮动体专用设置文件
% Double Column Float Specific Configuration
% =============================================================================
% 文件用途:双栏布局中浮动体(图表)的专用配置
% 创建日期:2025-08-03
% 版本:v1.0
% 
% 使用说明:
% 1. 本文件专门配置双栏布局中的浮动体行为
% 2. 在主文档的 preamble 中使用 \input{preamble/layout_double/floats.tex} 引入
% 3. 建议在引入 layout_double/settings.tex 之后引入本文件
% 4. 本文件解决双栏布局中图表放置的特殊需求
% 
% 配置内容包括:
% - 跨栏浮动体设置(figure*、table* 环境)
% - 单栏内浮动体位置控制
% - 浮动体与文本的间距调整
% - 图表标题在双栏中的格式设置
% - 浮动体排队和放置算法优化
% - 特殊浮动体环境定义
% 
% 注意事项:
% - 此配置仅适用于双栏布局
% - 跨栏浮动体会影响页面布局,需谨慎使用
% - 某些设置可能与栏平衡功能产生冲突
% - 建议配合 balance.tex 一起调试
% =============================================================================

% 此文件将在后续迁移过程中填充具体的双栏浮动体配置内容 引入
% 3. 建议在引入 layout_double/settings.tex 之后引入本文件
% 4. 本文件解决双栏布局中图表放置的特殊需求
% 
% 配置内容包括:
% - 跨栏浮动体设置(figure*、table* 环境)
% - 单栏内浮动体位置控制
% - 浮动体与文本的间距调整
% - 图表标题在双栏中的格式设置
% - 浮动体排队和放置算法优化
% - 特殊浮动体环境定义
% 
% 注意事项:
% - 此配置仅适用于双栏布局
% - 跨栏浮动体会影响页面布局,需谨慎使用
% - 某些设置可能与栏平衡功能产生冲突
% - 建议配合 balance.tex 一起调试
% =============================================================================

% 此文件将在后续迁移过程中填充具体的双栏浮动体配置内容 引入
% 3. 建议在引入 layout_double/settings.tex 之后引入本文件
% 4. 本文件解决双栏布局中图表放置的特殊需求
% 
% 配置内容包括:
% - 跨栏浮动体设置(figure*、table* 环境)
% - 单栏内浮动体位置控制
% - 浮动体与文本的间距调整
% - 图表标题在双栏中的格式设置
% - 浮动体排队和放置算法优化
% - 特殊浮动体环境定义
% 
% 注意事项:
% - 此配置仅适用于双栏布局
% - 跨栏浮动体会影响页面布局,需谨慎使用
% - 某些设置可能与栏平衡功能产生冲突
% - 建议配合 balance.tex 一起调试
% =============================================================================

% 此文件将在后续迁移过程中填充具体的双栏浮动体配置内容    % 浮动体优化
% =============================================================================
% 栏平衡功能配置文件
% Column Balance Feature Configuration
% =============================================================================
% 文件用途:双栏布局中的栏平衡功能配置
% 创建日期:2025-08-03
% 版本:v1.0
% 
% 使用说明:
% 1. 本文件提供双栏布局中的栏平衡功能
% 2. 在主文档的 preamble 中使用 % =============================================================================
% 栏平衡功能配置文件
% Column Balance Feature Configuration
% =============================================================================
% 文件用途:双栏布局中的栏平衡功能配置
% 创建日期:2025-08-03
% 版本:v1.0
% 
% 使用说明:
% 1. 本文件提供双栏布局中的栏平衡功能
% 2. 在主文档的 preamble 中使用 % =============================================================================
% 栏平衡功能配置文件
% Column Balance Feature Configuration
% =============================================================================
% 文件用途:双栏布局中的栏平衡功能配置
% 创建日期:2025-08-03
% 版本:v1.0
% 
% 使用说明:
% 1. 本文件提供双栏布局中的栏平衡功能
% 2. 在主文档的 preamble 中使用 \input{preamble/layout_double/balance.tex} 引入
% 3. 建议在引入其他双栏配置文件之后引入本文件
% 4. 栏平衡可确保双栏页面左右栏高度尽可能相等
% 
% 配置内容包括:
% - 自动栏平衡设置
% - 手动栏平衡命令定义
% - 栏平衡的触发条件配置
% - 最后一页的栏平衡处理
% - 栏平衡与浮动体的协调
% - 栏平衡算法的参数调整
% 
% 常用命令(将在配置中定义):
% - \balancecolumns : 强制当前页面栏平衡
% - \nobalancecolumns : 取消自动栏平衡
% - \lastpagebalance : 最后一页栏平衡
% 
% 注意事项:
% - 栏平衡可能影响页面排版效果
% - 与某些浮动体设置可能产生冲突
% - 建议根据具体文档内容调整平衡策略
% - 在文档最终定稿前进行栏平衡调试
% =============================================================================

% ========== 栏平衡包导入 ==========
\usepackage{balance}        % 双栏平衡包
% \usepackage{multicol}     % 多栏支持包 - 已禁用,避免与 twocolumn 文档类冲突

% ========== 自动栏平衡设置 ==========

% --- 最后一页自动平衡 ---
% 默认启用最后一页的栏平衡(已移至文档结尾处理)
% \balance  % 禁用全局默认栏平衡,避免与 twocolumn 文档类冲突

% --- 栏平衡参数调整 ---
% 调整栏平衡的容差和算法参数
\makeatletter
% 设置栏平衡的最大迭代次数
\def\balance@max@iterations{10}

% 设置栏高差的容差(单位:pt)
\def\balance@tolerance{12pt}

% 设置栏平衡的惩罚值
\clubpenalty=10000          % 防止孤行
\widowpenalty=10000         % 防止寡行
\displaywidowpenalty=10000  % 防止公式后寡行
\makeatother

% ========== 手动栏平衡命令 ==========

% --- 基本栏平衡命令 ---
% 强制当前页面进行栏平衡
\newcommand{\balancecolumns}{%
    \vfill\eject
    \balance
}

% 取消自动栏平衡
\newcommand{\nobalancecolumns}{%
    \nobalance
}

% 恢复自动栏平衡
\newcommand{\restorebalance}{%
    \balance
}

% --- 条件栏平衡命令 ---
% 仅在最后一页进行栏平衡
\newcommand{\lastpagebalance}{%
    \AtEndDocument{\balance}
}

% 在指定页面进行栏平衡
\newcommand{\balancepage}[1]{%
    \AtBeginShipoutNext{\balance}
}

% ========== 栏平衡质量控制 ==========

% --- 栏间距调整 ---
% 在平衡时稍微增加栏间距以改善视觉效果
\newcommand{\adjustcolsepforbalance}{%
    \addtolength{\columnsep}{2pt}
}

% 恢复默认栏间距
\newcommand{\restorecolsep}{%
    \addtolength{\columnsep}{-2pt}
}

% --- 栏平衡质量检查 ---
% 检查当前页面是否需要栏平衡
\newcommand{\checkbalanceneed}{%
    \ifdim\pagetotal>\pagegoal
        \typeout{Warning: Page is overfull, consider manual balance}
    \fi
}

% ========== 与浮动体的协调 ==========

% --- 浮动体感知的栏平衡 ---
% 在有浮动体时更谨慎地进行栏平衡
\newcommand{\balancewithfloats}{%
    \clearpage  % 先处理所有浮动体
    \balance    % 然后进行栏平衡
}

% --- 跨栏浮动体后的平衡处理 ---
% 跨栏浮动体可能影响栏平衡,需要特殊处理
\newcommand{\balanceafterfigstar}{%
    \vspace{0pt plus 1fil}  % 添加弹性垂直空间
    \balance
}

% ========== 特殊情况处理 ==========

% --- 短页面栏平衡 ---
% 对于内容较少的页面,可能不需要强制平衡
\newcommand{\smartbalance}{%
    \ifdim\pagetotal<0.7\pagegoal
        \balance
    \else
        \typeout{Page too full for smart balance}
    \fi
}

% --- 段落感知的栏平衡 ---
% 避免在段落中间进行栏平衡
\newcommand{\paragraphawarebalance}{%
    \par
    \vspace{0pt plus 1fil}
    \balance
}

% ========== 栏平衡调试工具 ==========

% --- 显示栏平衡状态 ---
\newcommand{\showbalancestatus}{%
    \typeout{栏平衡状态信息:}%
    \typeout{  当前页面填充度: \the\pagetotal}%
    \typeout{  页面目标高度: \the\pagegoal}%
    \typeout{  栏间距: \the\columnsep}%
    \typeout{  平衡状态: \ifbalance@enabled{启用}\else{禁用}\fi}%
}

% --- 可视化栏平衡 ---
% 在调试时显示栏的边界
\newif\ifshowcolumns
\showcolumnsfalse  % 默认关闭

\newcommand{\showcolumnborders}{%
    \showcolumnstrue
    \makeatletter
    \def\columnseprule{0.1pt}
    \makeatother
}

\newcommand{\hidecolumnborders}{%
    \showcolumnsfalse
    \makeatletter
    \def\columnseprule{0pt}
    \makeatother
}

% ========== 高级栏平衡功能 ==========

% --- 渐进式栏平衡 ---
% 逐步调整栏平衡,避免突然的变化
\newcommand{\progressivebalance}{%
    \vspace{0pt plus 0.5fil}
    \balance
    \vspace{0pt plus 0.5fil}
}

% --- 内容感知平衡 ---
% 根据内容类型选择平衡策略
\newcommand{\contentawarebalance}[1]{%
    \if\relax\detokenize{#1}\relax
        % 无参数,使用默认平衡
        \balance
    \else
        \ifx#1\figure
            % 图片后的平衡
            \vspace{6pt plus 2pt minus 2pt}
            \balance
        \else\ifx#1\table
            % 表格后的平衡
            \vspace{4pt plus 2pt minus 1pt}
            \balance
        \else\ifx#1\equation
            % 公式后的平衡
            \vspace{3pt plus 1pt minus 1pt}
            \balance
        \else
            % 其他内容的默认平衡
            \balance
        \fi\fi\fi
    \fi
}

% ========== 栏平衡的全局策略 ==========

% --- 文档级栏平衡设置 ---
% 在文档开始时设置全局栏平衡策略
% 注意:禁用文档开始时的自动栏平衡,避免与原生 twocolumn 冲突
\AtBeginDocument{%
    % \balance  % 禁用自动栏平衡,改为仅在文档结尾启用
    \typeout{双栏布局栏平衡功能已配置(仅文档结尾生效)}
}

% --- 章节级栏平衡 ---
% 在每个章节开始时重置栏平衡
% 注意:为避免与已加载包的冲突,暂时禁用自动章节栏平衡
% 用户可手动调用 \balancecolumns 命令进行栏平衡
% \let\orig@section\section
% \renewcommand{\section}[1]{%
%     \balance
%     \orig@section{#1}
% }

% --- 文档结束时的栏平衡 ---
\AtEndDocument{%
    \balance
    \typeout{文档结束,最终栏平衡完成}
} 引入
% 3. 建议在引入其他双栏配置文件之后引入本文件
% 4. 栏平衡可确保双栏页面左右栏高度尽可能相等
% 
% 配置内容包括:
% - 自动栏平衡设置
% - 手动栏平衡命令定义
% - 栏平衡的触发条件配置
% - 最后一页的栏平衡处理
% - 栏平衡与浮动体的协调
% - 栏平衡算法的参数调整
% 
% 常用命令(将在配置中定义):
% - \balancecolumns : 强制当前页面栏平衡
% - \nobalancecolumns : 取消自动栏平衡
% - \lastpagebalance : 最后一页栏平衡
% 
% 注意事项:
% - 栏平衡可能影响页面排版效果
% - 与某些浮动体设置可能产生冲突
% - 建议根据具体文档内容调整平衡策略
% - 在文档最终定稿前进行栏平衡调试
% =============================================================================

% ========== 栏平衡包导入 ==========
\usepackage{balance}        % 双栏平衡包
% \usepackage{multicol}     % 多栏支持包 - 已禁用,避免与 twocolumn 文档类冲突

% ========== 自动栏平衡设置 ==========

% --- 最后一页自动平衡 ---
% 默认启用最后一页的栏平衡(已移至文档结尾处理)
% \balance  % 禁用全局默认栏平衡,避免与 twocolumn 文档类冲突

% --- 栏平衡参数调整 ---
% 调整栏平衡的容差和算法参数
\makeatletter
% 设置栏平衡的最大迭代次数
\def\balance@max@iterations{10}

% 设置栏高差的容差(单位:pt)
\def\balance@tolerance{12pt}

% 设置栏平衡的惩罚值
\clubpenalty=10000          % 防止孤行
\widowpenalty=10000         % 防止寡行
\displaywidowpenalty=10000  % 防止公式后寡行
\makeatother

% ========== 手动栏平衡命令 ==========

% --- 基本栏平衡命令 ---
% 强制当前页面进行栏平衡
\newcommand{\balancecolumns}{%
    \vfill\eject
    \balance
}

% 取消自动栏平衡
\newcommand{\nobalancecolumns}{%
    \nobalance
}

% 恢复自动栏平衡
\newcommand{\restorebalance}{%
    \balance
}

% --- 条件栏平衡命令 ---
% 仅在最后一页进行栏平衡
\newcommand{\lastpagebalance}{%
    \AtEndDocument{\balance}
}

% 在指定页面进行栏平衡
\newcommand{\balancepage}[1]{%
    \AtBeginShipoutNext{\balance}
}

% ========== 栏平衡质量控制 ==========

% --- 栏间距调整 ---
% 在平衡时稍微增加栏间距以改善视觉效果
\newcommand{\adjustcolsepforbalance}{%
    \addtolength{\columnsep}{2pt}
}

% 恢复默认栏间距
\newcommand{\restorecolsep}{%
    \addtolength{\columnsep}{-2pt}
}

% --- 栏平衡质量检查 ---
% 检查当前页面是否需要栏平衡
\newcommand{\checkbalanceneed}{%
    \ifdim\pagetotal>\pagegoal
        \typeout{Warning: Page is overfull, consider manual balance}
    \fi
}

% ========== 与浮动体的协调 ==========

% --- 浮动体感知的栏平衡 ---
% 在有浮动体时更谨慎地进行栏平衡
\newcommand{\balancewithfloats}{%
    \clearpage  % 先处理所有浮动体
    \balance    % 然后进行栏平衡
}

% --- 跨栏浮动体后的平衡处理 ---
% 跨栏浮动体可能影响栏平衡,需要特殊处理
\newcommand{\balanceafterfigstar}{%
    \vspace{0pt plus 1fil}  % 添加弹性垂直空间
    \balance
}

% ========== 特殊情况处理 ==========

% --- 短页面栏平衡 ---
% 对于内容较少的页面,可能不需要强制平衡
\newcommand{\smartbalance}{%
    \ifdim\pagetotal<0.7\pagegoal
        \balance
    \else
        \typeout{Page too full for smart balance}
    \fi
}

% --- 段落感知的栏平衡 ---
% 避免在段落中间进行栏平衡
\newcommand{\paragraphawarebalance}{%
    \par
    \vspace{0pt plus 1fil}
    \balance
}

% ========== 栏平衡调试工具 ==========

% --- 显示栏平衡状态 ---
\newcommand{\showbalancestatus}{%
    \typeout{栏平衡状态信息:}%
    \typeout{  当前页面填充度: \the\pagetotal}%
    \typeout{  页面目标高度: \the\pagegoal}%
    \typeout{  栏间距: \the\columnsep}%
    \typeout{  平衡状态: \ifbalance@enabled{启用}\else{禁用}\fi}%
}

% --- 可视化栏平衡 ---
% 在调试时显示栏的边界
\newif\ifshowcolumns
\showcolumnsfalse  % 默认关闭

\newcommand{\showcolumnborders}{%
    \showcolumnstrue
    \makeatletter
    \def\columnseprule{0.1pt}
    \makeatother
}

\newcommand{\hidecolumnborders}{%
    \showcolumnsfalse
    \makeatletter
    \def\columnseprule{0pt}
    \makeatother
}

% ========== 高级栏平衡功能 ==========

% --- 渐进式栏平衡 ---
% 逐步调整栏平衡,避免突然的变化
\newcommand{\progressivebalance}{%
    \vspace{0pt plus 0.5fil}
    \balance
    \vspace{0pt plus 0.5fil}
}

% --- 内容感知平衡 ---
% 根据内容类型选择平衡策略
\newcommand{\contentawarebalance}[1]{%
    \if\relax\detokenize{#1}\relax
        % 无参数,使用默认平衡
        \balance
    \else
        \ifx#1\figure
            % 图片后的平衡
            \vspace{6pt plus 2pt minus 2pt}
            \balance
        \else\ifx#1\table
            % 表格后的平衡
            \vspace{4pt plus 2pt minus 1pt}
            \balance
        \else\ifx#1\equation
            % 公式后的平衡
            \vspace{3pt plus 1pt minus 1pt}
            \balance
        \else
            % 其他内容的默认平衡
            \balance
        \fi\fi\fi
    \fi
}

% ========== 栏平衡的全局策略 ==========

% --- 文档级栏平衡设置 ---
% 在文档开始时设置全局栏平衡策略
% 注意:禁用文档开始时的自动栏平衡,避免与原生 twocolumn 冲突
\AtBeginDocument{%
    % \balance  % 禁用自动栏平衡,改为仅在文档结尾启用
    \typeout{双栏布局栏平衡功能已配置(仅文档结尾生效)}
}

% --- 章节级栏平衡 ---
% 在每个章节开始时重置栏平衡
% 注意:为避免与已加载包的冲突,暂时禁用自动章节栏平衡
% 用户可手动调用 \balancecolumns 命令进行栏平衡
% \let\orig@section\section
% \renewcommand{\section}[1]{%
%     \balance
%     \orig@section{#1}
% }

% --- 文档结束时的栏平衡 ---
\AtEndDocument{%
    \balance
    \typeout{文档结束,最终栏平衡完成}
} 引入
% 3. 建议在引入其他双栏配置文件之后引入本文件
% 4. 栏平衡可确保双栏页面左右栏高度尽可能相等
% 
% 配置内容包括:
% - 自动栏平衡设置
% - 手动栏平衡命令定义
% - 栏平衡的触发条件配置
% - 最后一页的栏平衡处理
% - 栏平衡与浮动体的协调
% - 栏平衡算法的参数调整
% 
% 常用命令(将在配置中定义):
% - \balancecolumns : 强制当前页面栏平衡
% - \nobalancecolumns : 取消自动栏平衡
% - \lastpagebalance : 最后一页栏平衡
% 
% 注意事项:
% - 栏平衡可能影响页面排版效果
% - 与某些浮动体设置可能产生冲突
% - 建议根据具体文档内容调整平衡策略
% - 在文档最终定稿前进行栏平衡调试
% =============================================================================

% ========== 栏平衡包导入 ==========
\usepackage{balance}        % 双栏平衡包
% \usepackage{multicol}     % 多栏支持包 - 已禁用,避免与 twocolumn 文档类冲突

% ========== 自动栏平衡设置 ==========

% --- 最后一页自动平衡 ---
% 默认启用最后一页的栏平衡(已移至文档结尾处理)
% \balance  % 禁用全局默认栏平衡,避免与 twocolumn 文档类冲突

% --- 栏平衡参数调整 ---
% 调整栏平衡的容差和算法参数
\makeatletter
% 设置栏平衡的最大迭代次数
\def\balance@max@iterations{10}

% 设置栏高差的容差(单位:pt)
\def\balance@tolerance{12pt}

% 设置栏平衡的惩罚值
\clubpenalty=10000          % 防止孤行
\widowpenalty=10000         % 防止寡行
\displaywidowpenalty=10000  % 防止公式后寡行
\makeatother

% ========== 手动栏平衡命令 ==========

% --- 基本栏平衡命令 ---
% 强制当前页面进行栏平衡
\newcommand{\balancecolumns}{%
    \vfill\eject
    \balance
}

% 取消自动栏平衡
\newcommand{\nobalancecolumns}{%
    \nobalance
}

% 恢复自动栏平衡
\newcommand{\restorebalance}{%
    \balance
}

% --- 条件栏平衡命令 ---
% 仅在最后一页进行栏平衡
\newcommand{\lastpagebalance}{%
    \AtEndDocument{\balance}
}

% 在指定页面进行栏平衡
\newcommand{\balancepage}[1]{%
    \AtBeginShipoutNext{\balance}
}

% ========== 栏平衡质量控制 ==========

% --- 栏间距调整 ---
% 在平衡时稍微增加栏间距以改善视觉效果
\newcommand{\adjustcolsepforbalance}{%
    \addtolength{\columnsep}{2pt}
}

% 恢复默认栏间距
\newcommand{\restorecolsep}{%
    \addtolength{\columnsep}{-2pt}
}

% --- 栏平衡质量检查 ---
% 检查当前页面是否需要栏平衡
\newcommand{\checkbalanceneed}{%
    \ifdim\pagetotal>\pagegoal
        \typeout{Warning: Page is overfull, consider manual balance}
    \fi
}

% ========== 与浮动体的协调 ==========

% --- 浮动体感知的栏平衡 ---
% 在有浮动体时更谨慎地进行栏平衡
\newcommand{\balancewithfloats}{%
    \clearpage  % 先处理所有浮动体
    \balance    % 然后进行栏平衡
}

% --- 跨栏浮动体后的平衡处理 ---
% 跨栏浮动体可能影响栏平衡,需要特殊处理
\newcommand{\balanceafterfigstar}{%
    \vspace{0pt plus 1fil}  % 添加弹性垂直空间
    \balance
}

% ========== 特殊情况处理 ==========

% --- 短页面栏平衡 ---
% 对于内容较少的页面,可能不需要强制平衡
\newcommand{\smartbalance}{%
    \ifdim\pagetotal<0.7\pagegoal
        \balance
    \else
        \typeout{Page too full for smart balance}
    \fi
}

% --- 段落感知的栏平衡 ---
% 避免在段落中间进行栏平衡
\newcommand{\paragraphawarebalance}{%
    \par
    \vspace{0pt plus 1fil}
    \balance
}

% ========== 栏平衡调试工具 ==========

% --- 显示栏平衡状态 ---
\newcommand{\showbalancestatus}{%
    \typeout{栏平衡状态信息:}%
    \typeout{  当前页面填充度: \the\pagetotal}%
    \typeout{  页面目标高度: \the\pagegoal}%
    \typeout{  栏间距: \the\columnsep}%
    \typeout{  平衡状态: \ifbalance@enabled{启用}\else{禁用}\fi}%
}

% --- 可视化栏平衡 ---
% 在调试时显示栏的边界
\newif\ifshowcolumns
\showcolumnsfalse  % 默认关闭

\newcommand{\showcolumnborders}{%
    \showcolumnstrue
    \makeatletter
    \def\columnseprule{0.1pt}
    \makeatother
}

\newcommand{\hidecolumnborders}{%
    \showcolumnsfalse
    \makeatletter
    \def\columnseprule{0pt}
    \makeatother
}

% ========== 高级栏平衡功能 ==========

% --- 渐进式栏平衡 ---
% 逐步调整栏平衡,避免突然的变化
\newcommand{\progressivebalance}{%
    \vspace{0pt plus 0.5fil}
    \balance
    \vspace{0pt plus 0.5fil}
}

% --- 内容感知平衡 ---
% 根据内容类型选择平衡策略
\newcommand{\contentawarebalance}[1]{%
    \if\relax\detokenize{#1}\relax
        % 无参数,使用默认平衡
        \balance
    \else
        \ifx#1\figure
            % 图片后的平衡
            \vspace{6pt plus 2pt minus 2pt}
            \balance
        \else\ifx#1\table
            % 表格后的平衡
            \vspace{4pt plus 2pt minus 1pt}
            \balance
        \else\ifx#1\equation
            % 公式后的平衡
            \vspace{3pt plus 1pt minus 1pt}
            \balance
        \else
            % 其他内容的默认平衡
            \balance
        \fi\fi\fi
    \fi
}

% ========== 栏平衡的全局策略 ==========

% --- 文档级栏平衡设置 ---
% 在文档开始时设置全局栏平衡策略
% 注意:禁用文档开始时的自动栏平衡,避免与原生 twocolumn 冲突
\AtBeginDocument{%
    % \balance  % 禁用自动栏平衡,改为仅在文档结尾启用
    \typeout{双栏布局栏平衡功能已配置(仅文档结尾生效)}
}

% --- 章节级栏平衡 ---
% 在每个章节开始时重置栏平衡
% 注意:为避免与已加载包的冲突,暂时禁用自动章节栏平衡
% 用户可手动调用 \balancecolumns 命令进行栏平衡
% \let\orig@section\section
% \renewcommand{\section}[1]{%
%     \balance
%     \orig@section{#1}
% }

% --- 文档结束时的栏平衡 ---
\AtEndDocument{%
    \balance
    \typeout{文档结束,最终栏平衡完成}
}   % 栏平衡优化

% --- 双栏专用命令 ---
% 跨栏注意框
\newcommand{\noteboxwide}[1]{% 
    \begin{figure*}[!b]
    \centering
    \fcolorbox{red}{yellow!20}{% 
        \parbox{0.9\textwidth}{% 
            \textbf{注意:}#1 
        } 
    } 
    \end{figure*}
}

% 双栏代码输入命令
\newcommand{\inputcodewide}[3][python]{% 
    \begin{figure*}[!t] 
        \inputminted{#1}{#2} 
        \caption{#3} 
    \end{figure*}
}