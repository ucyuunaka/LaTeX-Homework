% =============================================================================
%                    双栏布局配置 - Double Column Layout
% =============================================================================

% ========== 文档类设置 ==========
\documentclass[10pt,twocolumn]{ctexart}

% ========== 引入通用配置 ==========
% =============================================================================
% 通用包导入配置文件
% Common Package Import Configuration
% =============================================================================
% 文件用途:存放所有通用的 LaTeX 宏包导入命令
% 创建日期:2025-08-03
% 版本:v1.0
% 
% 使用说明:
% 1. 本文件包含项目中所有通用的宏包导入
% 2. 在主文档的 preamble 中使用 % =============================================================================
% 通用包导入配置文件
% Common Package Import Configuration
% =============================================================================
% 文件用途:存放所有通用的 LaTeX 宏包导入命令
% 创建日期:2025-08-03
% 版本:v1.0
% 
% 使用说明:
% 1. 本文件包含项目中所有通用的宏包导入
% 2. 在主文档的 preamble 中使用 % =============================================================================
% 通用包导入配置文件
% Common Package Import Configuration
% =============================================================================
% 文件用途:存放所有通用的 LaTeX 宏包导入命令
% 创建日期:2025-08-03
% 版本:v1.0
% 
% 使用说明:
% 1. 本文件包含项目中所有通用的宏包导入
% 2. 在主文档的 preamble 中使用 \input{preamble/common/packages.tex} 引入
% 3. 按照功能分类组织宏包导入,便于维护和管理
% 4. 包含详细的注释说明每个宏包的用途
% 
% 注意事项:
% - 此文件应该只包含通用宏包,特定布局的宏包请放在对应的布局配置中
% - 宏包导入顺序可能影响功能,请谨慎调整
% =============================================================================

% 此文件将在后续迁移过程中填充具体的宏包导入内容 引入
% 3. 按照功能分类组织宏包导入,便于维护和管理
% 4. 包含详细的注释说明每个宏包的用途
% 
% 注意事项:
% - 此文件应该只包含通用宏包,特定布局的宏包请放在对应的布局配置中
% - 宏包导入顺序可能影响功能,请谨慎调整
% =============================================================================

% 此文件将在后续迁移过程中填充具体的宏包导入内容 引入
% 3. 按照功能分类组织宏包导入,便于维护和管理
% 4. 包含详细的注释说明每个宏包的用途
% 
% 注意事项:
% - 此文件应该只包含通用宏包,特定布局的宏包请放在对应的布局配置中
% - 宏包导入顺序可能影响功能,请谨慎调整
% =============================================================================

% 此文件将在后续迁移过程中填充具体的宏包导入内容
% =============================================================================
%                    通用命令与环境定义 - Common Commands
% =============================================================================

% ========== 定理环境设置 ==========
\theoremstyle{definition} 
\newtheorem{definition}{定义}[section] 
\newtheorem{theorem}{定理}[section] 
\newtheorem{lemma}{引理}[section] 
\newtheorem{corollary}{推论}[section] 
\newtheorem{example}{例}[section] 
\newtheorem{remark}{注}[section] 

% ========== 链接样式设置 ==========
\hypersetup{colorlinks=true,linkcolor=black,citecolor=green,urlcolor=blue} 

% ========== 代码样式配置 ==========
\definecolor{codebg}{rgb}{0.95,0.95,0.95} 
\definecolor{commentcolor}{rgb}{0.0,0.5,0.0} 
\definecolor{keywordcolor}{rgb}{0.0,0.0,0.8} 
\definecolor{stringcolor}{rgb}{0.8,0.0,0.0} 

\setminted{
    bgcolor=codebg,
    linenos,
    breaklines,
    breakanywhere,
    fontsize=\small,
    frame=single,
    framesep=2mm
} 

\lstset{
    backgroundcolor=\color{codebg},
    basicstyle=\ttfamily\small,
    breakatwhitespace=false,
    breaklines=true,
    captionpos=b,
    commentstyle=\color{commentcolor}\itshape,
    escapeinside={\%*}{*)},
    extendedchars=true,
    frame=single,
    keepspaces=true,
    keywordstyle=\color{keywordcolor}\bfseries,
    language=Python,
    numbers=left, 
    numbersep=5pt, 
    numberstyle=\tiny\color{gray}, 
    rulecolor=\color{black}, 
    showspaces=false, 
    showstringspaces=false, 
    showtabs=false, 
    stepnumber=1, 
    stringstyle=\color{stringcolor}, 
    tabsize=4, 
    title=\lstname, 
    postbreak=\mbox{\textcolor{red}{$\hookrightarrow$}\space}, 
    xleftmargin=2em, 
    framexleftmargin=1.5em 
}

\lstdefinelanguage{JavaScript}{
    keywords={break, case, catch, continue, debugger, default, delete, do, else, false, finally, for, function, if, in, instanceof, new, null, return, switch, this, throw, true, try, typeof, var, void, while, with, const, let, class, export, import, async, await}, 
    morecomment=[l]{//}, 
    morecomment=[s]{/*}{*/}, 
    morestring=[b]', 
    morestring=[b]", 
    morestring=[b]`, 
    ndkeywords={class, export, boolean, throw, implements, import, this}, 
    keywordstyle=\color{keywordcolor}\bfseries, 
    ndkeywordstyle=\color{purple}\bfseries, 
    identifierstyle=\color{black}, 
    commentstyle=\color{commentcolor}\ttfamily, 
    stringstyle=\color{stringcolor}\ttfamily, 
    sensitive=true 
}

\SetAlgorithmName{算法}{算法}{算法列表} 
\SetKwProg{Fn}{函数}{:}{结束} 
\SetKwFunction{FMain}{主函数} 

\captionsetup{
    font={small,bf},
    labelfont=bf,
    textfont=rm,
    justification=centering
} 

\setlist[itemize]{leftmargin=2em,itemsep=0.5ex} 
\setlist[enumerate]{leftmargin=2em,itemsep=0.5ex} 

% ========== 自定义命令 ==========
\newcommand{\highlight}[1]{\colorbox{yellow}{#1}} 

\newcommand{\notebox}[1]{% 
    \begin{center} 
    \fcolorbox{red}{yellow!20}{% 
        \parbox{0.9\textwidth}{% 
            \textbf{注意:}#1 
        } 
    } 
    \end{center} 
}

\newcommand{\insertfig}[4][0.8]{% 
    \begin{figure}[H] 
        \centering 
        \includegraphics[width=#1\textwidth]{#2} 
        \caption{#3} 
        \label{#4} 
    \end{figure} 
}

\newcommand{\inputcode}[3][python]{% 
    \begin{listing}[H] 
        \inputminted{#1}{#2} 
        \caption{#3} 
    \end{listing} 
}

\makeatletter 
\newcommand\dlmu[2][4cm]{% 
  \sbox0{#2}% 
  \ifdim\wd0>#1\relax 
    \hskip1pt\underline{\hb@xt@ #1{\hss\resizebox*{#1}{!}{\mbox{#2}}\hss}}\hskip3pt% 
  \else 
    \hskip1pt\underline{\hb@xt@ #1{\hss#2\hss}}\hskip3pt% 
  \fi 
}
\makeatother

% ========== 双栏布局特定设置 ==========

% --- 页面格式设置 ---
\setlength{\parindent}{1.5em}  % 双栏布局缩进稍小
\geometry{
    left=1.5cm,
    right=1.5cm,
    top=2cm,
    bottom=2cm,
    headsep=1cm,
    columnsep=0.8cm  % 栏间距
} 
\pagestyle{fancy}
\setlength{\headheight}{33pt}  % 修正页眉高度以避免警告
\renewcommand{\headrulewidth}{0.4pt}
\renewcommand{\footrulewidth}{0pt}
\setstretch{1.2}  % 双栏布局行距紧凑

% --- 双栏图片尺寸优化 ---
% 重新定义 insertfig 命令以适应双栏布局
\renewcommand{\insertfig}[4][0.9]{%
    \begin{figure}[H] 
        \centering 
        \includegraphics[width=#1\columnwidth]{#2} 
        \caption{#3} 
        \label{#4} 
    \end{figure} 
}

% 定义跨栏图片命令
\newcommand{\insertfigwide}[4][0.8]{%
    \begin{figure*}[!t] 
        \centering 
        \includegraphics[width=#1\textwidth]{#2} 
        \caption{#3} 
        \label{#4} 
    \end{figure*} 
}

% --- 双栏代码块优化 ---
% 调整代码块样式以适应双栏
\setminted{
    bgcolor=codebg,
    linenos=false,  % 双栏时关闭行号节省空间
    breaklines,
    breakanywhere,
    fontsize=\footnotesize,  % 字体更小
    frame=single,
    framesep=1mm  % 边距更小
}

\lstset{
    backgroundcolor=\color{codebg},
    basicstyle=\ttfamily\footnotesize,  % 字体更小
    breakatwhitespace=false,
    breaklines=true,
    captionpos=b,
    commentstyle=\color{commentcolor}\itshape,
    escapeinside={\%*}{*)},
    extendedchars=true,
    frame=single,
    keepspaces=true,
    keywordstyle=\color{keywordcolor}\bfseries,
    language=Python,
    numbers=none,  % 双栏时关闭行号
    rulecolor=\color{black}, 
    showspaces=false, 
    showstringspaces=false, 
    showtabs=false, 
    stepnumber=1, 
    stringstyle=\color{stringcolor}, 
    tabsize=4, 
    title=\lstname, 
    postbreak=\mbox{\textcolor{red}{$\hookrightarrow$}\space}, 
    xleftmargin=1em,  % 左边距更小
    framexleftmargin=0.5em 
}

% --- 双栏表格优化 ---
% 调整标题样式以适应双栏
\captionsetup{
    font={footnotesize,bf},  % 字体更小
    labelfont=bf,
    textfont=rm,
    justification=centering
}

% --- 双栏列表优化 ---
\setlist[itemize]{leftmargin=1.5em,itemsep=0.3ex}  % 间距更紧凑
\setlist[enumerate]{leftmargin=1.5em,itemsep=0.3ex}

% --- 引入双栏专用模块 ---
% =============================================================================
% 双栏浮动体专用设置文件
% Double Column Float Specific Configuration
% =============================================================================
% 文件用途:双栏布局中浮动体(图表)的专用配置
% 使用说明:
% 1. 本文件专门配置双栏布局中的浮动体行为
% 2. 在主文档的 preamble 中使用 % =============================================================================
% 双栏浮动体专用设置文件
% Double Column Float Specific Configuration
% =============================================================================
% 文件用途:双栏布局中浮动体(图表)的专用配置
% 使用说明:
% 1. 本文件专门配置双栏布局中的浮动体行为
% 2. 在主文档的 preamble 中使用 % =============================================================================
% 双栏浮动体专用设置文件
% Double Column Float Specific Configuration
% =============================================================================
% 文件用途:双栏布局中浮动体(图表)的专用配置
% 使用说明:
% 1. 本文件专门配置双栏布局中的浮动体行为
% 2. 在主文档的 preamble 中使用 \input{preamble/layout_double/floats.tex} 引入
% 3. 建议在引入 layout_double/settings.tex 之后引入本文件
% 4. 本文件解决双栏布局中图表放置的特殊需求
% 
% 配置内容包括:
% - 跨栏浮动体设置(figure*、table* 环境)
% - 单栏内浮动体位置控制
% - 浮动体与文本的间距调整
% - 图表标题在双栏中的格式设置
% - 浮动体排队和放置算法优化
% - 特殊浮动体环境定义
% 
% 注意事项:
% - 此配置仅适用于双栏布局
% - 跨栏浮动体会影响页面布局,需谨慎使用
% - 某些设置可能与栏平衡功能产生冲突
% - 建议配合 balance.tex 一起调试
% =============================================================================

% ========== 双栏浮动体参数设置 ==========

% --- 浮动体放置策略 ---
% 设置浮动体的默认放置参数
\renewcommand{\topfraction}{0.9}        % 页面顶部浮动体最大占比
\renewcommand{\bottomfraction}{0.8}     % 页面底部浮动体最大占比
\renewcommand{\textfraction}{0.1}       % 页面文本最小占比
\renewcommand{\floatpagefraction}{0.8}  % 浮动页浮动体最小占比
\setcounter{topnumber}{3}               % 页面顶部最多浮动体数量
\setcounter{bottomnumber}{2}            % 页面底部最多浮动体数量
\setcounter{totalnumber}{4}             % 每页最多浮动体总数

% --- 双栏内浮动体设置 ---
% 图片默认浮动策略:就近放置
\makeatletter
\renewcommand{\fps@figure}{!htb}        % 图片浮动顺序:强制here, top, bottom
\renewcommand{\fps@table}{!htb}         % 表格浮动顺序:强制here, top, bottom
\makeatother

% --- 跨栏浮动体设置 ---
% 跨栏图片和表格的默认设置
% 注意:在某些情况下这些命令可能未定义,需要条件检查
\makeatletter
\@ifundefined{fps@figure*}{}{%
    \renewcommand{\fps@figure*}{!tb}        % 跨栏图片:top, bottom(不允许here)
}
\@ifundefined{fps@table*}{}{%
    \renewcommand{\fps@table*}{!tb}         % 跨栏表格:top, bottom(不允许here)
}
\makeatother

% --- 浮动体间距调整 ---
% 浮动体与文本之间的间距(双栏布局需要更紧凑)
\setlength{\floatsep}{8pt plus 2pt minus 2pt}        % 同一页面浮动体间距
\setlength{\textfloatsep}{12pt plus 2pt minus 4pt}   % 浮动体与文本间距
\setlength{\intextsep}{8pt plus 2pt minus 2pt}       % 文中浮动体间距
\setlength{\dblfloatsep}{8pt plus 2pt minus 2pt}     % 跨栏浮动体间距
\setlength{\dbltextfloatsep}{12pt plus 2pt minus 4pt} % 跨栏浮动体与文本间距

% ========== 浮动体环境优化 ==========

% --- 双栏内小图片环境 ---
\newcommand{\insertfigsmall}[4][0.7]{%
    \begin{figure}[!htb]
        \centering
        \includegraphics[width=#1\columnwidth]{#2}
        \caption{#3}
        \label{#4}
    \end{figure}
}

% --- 双栏内紧凑表格环境 ---
\newcommand{\inserttablecompact}[3]{%
    \begin{table}[!htb]
        \centering
        \footnotesize  % 使用小字体
        #1
        \caption{#2}
        \label{#3}
    \end{table}
}

% --- 跨栏大表格环境 ---
\newcommand{\inserttablewide}[3]{%
    \begin{table*}[!tb]
        \centering
        \small  % 使用小字体
        #1
        \caption{#2}
        \label{#3}
    \end{table*}
}

% --- 并排子图环境(双栏内) ---
\newcommand{\insertsubfigs}[9][0.45]{%
    \begin{figure}[!htb]
        \centering
        \begin{subfigure}{#1\columnwidth}
            \centering
            \includegraphics[width=\textwidth]{#2}
            \caption{#3}
            \label{#4}
        \end{subfigure}
        \hfill
        \begin{subfigure}{#1\columnwidth}
            \centering
            \includegraphics[width=\textwidth]{#5}
            \caption{#6}
            \label{#7}
        \end{subfigure}
        \caption{#8}
        \label{#9}
    \end{figure}
}

% ========== 浮动体排版优化 ==========

% --- 图表标题格式 ---
% 双栏布局中的标题格式调整
\captionsetup[figure]{
    font=footnotesize,
    labelfont=bf,
    textfont=rm,
    justification=centering,
    singlelinecheck=false,
    skip=6pt
}

\captionsetup[table]{
    font=footnotesize,
    labelfont=bf,
    textfont=rm,
    justification=centering,
    singlelinecheck=false,
    skip=6pt
}

% --- 跨栏图表标题格式 ---
\DeclareCaptionStyle{widefigure}{
    font=small,
    labelfont=bf,
    textfont=rm,
    justification=centering,
    singlelinecheck=false,
    skip=8pt
}

\captionsetup[figure*]{style=widefigure}
\captionsetup[table*]{style=widefigure}

% ========== 特殊浮动体处理 ==========

% --- 侧边注释框(仅限单栏内) ---
\newcommand{\sidenotebox}[1]{%
    \begin{figure}[!htb]
        \centering
        \fcolorbox{blue}{blue!10}{%
            \parbox{0.9\columnwidth}{%
                \footnotesize
                \textbf{提示:}#1
            }
        }
    \end{figure}
}

% --- 算法环境适配双栏 ---
\floatstyle{ruled}
\newfloat{algorithm}{!htb}{loa}
\floatname{algorithm}{算法}

% 双栏算法环境
\newcommand{\insertalgorithm}[3]{%
    \begin{algorithm}[!htb]
        \centering
        \footnotesize
        #1
        \caption{#2}
        \label{#3}
    \end{algorithm}
}

% 跨栏算法环境
\newfloat{algorithm*}{!tb}{loa}
\floatname{algorithm*}{算法}

\newcommand{\insertalgorithmwide}[3]{%
    \begin{algorithm*}[!tb]
        \centering
        \small
        #1
        \caption{#2}
        \label{#3}
    \end{algorithm*}
}

% ========== 浮动体调试命令 ==========
% 用于调试浮动体位置的命令
\newcommand{\showfloatstatus}{%
    \typeout{当前浮动体状态:}%
    \typeout{  topnumber: \the\c@topnumber}%
    \typeout{  bottomnumber: \the\c@bottomnumber}%
    \typeout{  totalnumber: \the\c@totalnumber}%
    \typeout{  topfraction: \topfraction}%
    \typeout{  bottomfraction: \bottomfraction}%
    \typeout{  textfraction: \textfraction}%
} 引入
% 3. 建议在引入 layout_double/settings.tex 之后引入本文件
% 4. 本文件解决双栏布局中图表放置的特殊需求
% 
% 配置内容包括:
% - 跨栏浮动体设置(figure*、table* 环境)
% - 单栏内浮动体位置控制
% - 浮动体与文本的间距调整
% - 图表标题在双栏中的格式设置
% - 浮动体排队和放置算法优化
% - 特殊浮动体环境定义
% 
% 注意事项:
% - 此配置仅适用于双栏布局
% - 跨栏浮动体会影响页面布局,需谨慎使用
% - 某些设置可能与栏平衡功能产生冲突
% - 建议配合 balance.tex 一起调试
% =============================================================================

% ========== 双栏浮动体参数设置 ==========

% --- 浮动体放置策略 ---
% 设置浮动体的默认放置参数
\renewcommand{\topfraction}{0.9}        % 页面顶部浮动体最大占比
\renewcommand{\bottomfraction}{0.8}     % 页面底部浮动体最大占比
\renewcommand{\textfraction}{0.1}       % 页面文本最小占比
\renewcommand{\floatpagefraction}{0.8}  % 浮动页浮动体最小占比
\setcounter{topnumber}{3}               % 页面顶部最多浮动体数量
\setcounter{bottomnumber}{2}            % 页面底部最多浮动体数量
\setcounter{totalnumber}{4}             % 每页最多浮动体总数

% --- 双栏内浮动体设置 ---
% 图片默认浮动策略:就近放置
\makeatletter
\renewcommand{\fps@figure}{!htb}        % 图片浮动顺序:强制here, top, bottom
\renewcommand{\fps@table}{!htb}         % 表格浮动顺序:强制here, top, bottom
\makeatother

% --- 跨栏浮动体设置 ---
% 跨栏图片和表格的默认设置
% 注意:在某些情况下这些命令可能未定义,需要条件检查
\makeatletter
\@ifundefined{fps@figure*}{}{%
    \renewcommand{\fps@figure*}{!tb}        % 跨栏图片:top, bottom(不允许here)
}
\@ifundefined{fps@table*}{}{%
    \renewcommand{\fps@table*}{!tb}         % 跨栏表格:top, bottom(不允许here)
}
\makeatother

% --- 浮动体间距调整 ---
% 浮动体与文本之间的间距(双栏布局需要更紧凑)
\setlength{\floatsep}{8pt plus 2pt minus 2pt}        % 同一页面浮动体间距
\setlength{\textfloatsep}{12pt plus 2pt minus 4pt}   % 浮动体与文本间距
\setlength{\intextsep}{8pt plus 2pt minus 2pt}       % 文中浮动体间距
\setlength{\dblfloatsep}{8pt plus 2pt minus 2pt}     % 跨栏浮动体间距
\setlength{\dbltextfloatsep}{12pt plus 2pt minus 4pt} % 跨栏浮动体与文本间距

% ========== 浮动体环境优化 ==========

% --- 双栏内小图片环境 ---
\newcommand{\insertfigsmall}[4][0.7]{%
    \begin{figure}[!htb]
        \centering
        \includegraphics[width=#1\columnwidth]{#2}
        \caption{#3}
        \label{#4}
    \end{figure}
}

% --- 双栏内紧凑表格环境 ---
\newcommand{\inserttablecompact}[3]{%
    \begin{table}[!htb]
        \centering
        \footnotesize  % 使用小字体
        #1
        \caption{#2}
        \label{#3}
    \end{table}
}

% --- 跨栏大表格环境 ---
\newcommand{\inserttablewide}[3]{%
    \begin{table*}[!tb]
        \centering
        \small  % 使用小字体
        #1
        \caption{#2}
        \label{#3}
    \end{table*}
}

% --- 并排子图环境(双栏内) ---
\newcommand{\insertsubfigs}[9][0.45]{%
    \begin{figure}[!htb]
        \centering
        \begin{subfigure}{#1\columnwidth}
            \centering
            \includegraphics[width=\textwidth]{#2}
            \caption{#3}
            \label{#4}
        \end{subfigure}
        \hfill
        \begin{subfigure}{#1\columnwidth}
            \centering
            \includegraphics[width=\textwidth]{#5}
            \caption{#6}
            \label{#7}
        \end{subfigure}
        \caption{#8}
        \label{#9}
    \end{figure}
}

% ========== 浮动体排版优化 ==========

% --- 图表标题格式 ---
% 双栏布局中的标题格式调整
\captionsetup[figure]{
    font=footnotesize,
    labelfont=bf,
    textfont=rm,
    justification=centering,
    singlelinecheck=false,
    skip=6pt
}

\captionsetup[table]{
    font=footnotesize,
    labelfont=bf,
    textfont=rm,
    justification=centering,
    singlelinecheck=false,
    skip=6pt
}

% --- 跨栏图表标题格式 ---
\DeclareCaptionStyle{widefigure}{
    font=small,
    labelfont=bf,
    textfont=rm,
    justification=centering,
    singlelinecheck=false,
    skip=8pt
}

\captionsetup[figure*]{style=widefigure}
\captionsetup[table*]{style=widefigure}

% ========== 特殊浮动体处理 ==========

% --- 侧边注释框(仅限单栏内) ---
\newcommand{\sidenotebox}[1]{%
    \begin{figure}[!htb]
        \centering
        \fcolorbox{blue}{blue!10}{%
            \parbox{0.9\columnwidth}{%
                \footnotesize
                \textbf{提示:}#1
            }
        }
    \end{figure}
}

% --- 算法环境适配双栏 ---
\floatstyle{ruled}
\newfloat{algorithm}{!htb}{loa}
\floatname{algorithm}{算法}

% 双栏算法环境
\newcommand{\insertalgorithm}[3]{%
    \begin{algorithm}[!htb]
        \centering
        \footnotesize
        #1
        \caption{#2}
        \label{#3}
    \end{algorithm}
}

% 跨栏算法环境
\newfloat{algorithm*}{!tb}{loa}
\floatname{algorithm*}{算法}

\newcommand{\insertalgorithmwide}[3]{%
    \begin{algorithm*}[!tb]
        \centering
        \small
        #1
        \caption{#2}
        \label{#3}
    \end{algorithm*}
}

% ========== 浮动体调试命令 ==========
% 用于调试浮动体位置的命令
\newcommand{\showfloatstatus}{%
    \typeout{当前浮动体状态:}%
    \typeout{  topnumber: \the\c@topnumber}%
    \typeout{  bottomnumber: \the\c@bottomnumber}%
    \typeout{  totalnumber: \the\c@totalnumber}%
    \typeout{  topfraction: \topfraction}%
    \typeout{  bottomfraction: \bottomfraction}%
    \typeout{  textfraction: \textfraction}%
} 引入
% 3. 建议在引入 layout_double/settings.tex 之后引入本文件
% 4. 本文件解决双栏布局中图表放置的特殊需求
% 
% 配置内容包括:
% - 跨栏浮动体设置(figure*、table* 环境)
% - 单栏内浮动体位置控制
% - 浮动体与文本的间距调整
% - 图表标题在双栏中的格式设置
% - 浮动体排队和放置算法优化
% - 特殊浮动体环境定义
% 
% 注意事项:
% - 此配置仅适用于双栏布局
% - 跨栏浮动体会影响页面布局,需谨慎使用
% - 某些设置可能与栏平衡功能产生冲突
% - 建议配合 balance.tex 一起调试
% =============================================================================

% ========== 双栏浮动体参数设置 ==========

% --- 浮动体放置策略 ---
% 设置浮动体的默认放置参数
\renewcommand{\topfraction}{0.9}        % 页面顶部浮动体最大占比
\renewcommand{\bottomfraction}{0.8}     % 页面底部浮动体最大占比
\renewcommand{\textfraction}{0.1}       % 页面文本最小占比
\renewcommand{\floatpagefraction}{0.8}  % 浮动页浮动体最小占比
\setcounter{topnumber}{3}               % 页面顶部最多浮动体数量
\setcounter{bottomnumber}{2}            % 页面底部最多浮动体数量
\setcounter{totalnumber}{4}             % 每页最多浮动体总数

% --- 双栏内浮动体设置 ---
% 图片默认浮动策略:就近放置
\makeatletter
\renewcommand{\fps@figure}{!htb}        % 图片浮动顺序:强制here, top, bottom
\renewcommand{\fps@table}{!htb}         % 表格浮动顺序:强制here, top, bottom
\makeatother

% --- 跨栏浮动体设置 ---
% 跨栏图片和表格的默认设置
% 注意:在某些情况下这些命令可能未定义,需要条件检查
\makeatletter
\@ifundefined{fps@figure*}{}{%
    \renewcommand{\fps@figure*}{!tb}        % 跨栏图片:top, bottom(不允许here)
}
\@ifundefined{fps@table*}{}{%
    \renewcommand{\fps@table*}{!tb}         % 跨栏表格:top, bottom(不允许here)
}
\makeatother

% --- 浮动体间距调整 ---
% 浮动体与文本之间的间距(双栏布局需要更紧凑)
\setlength{\floatsep}{8pt plus 2pt minus 2pt}        % 同一页面浮动体间距
\setlength{\textfloatsep}{12pt plus 2pt minus 4pt}   % 浮动体与文本间距
\setlength{\intextsep}{8pt plus 2pt minus 2pt}       % 文中浮动体间距
\setlength{\dblfloatsep}{8pt plus 2pt minus 2pt}     % 跨栏浮动体间距
\setlength{\dbltextfloatsep}{12pt plus 2pt minus 4pt} % 跨栏浮动体与文本间距

% ========== 浮动体环境优化 ==========

% --- 双栏内小图片环境 ---
\newcommand{\insertfigsmall}[4][0.7]{%
    \begin{figure}[!htb]
        \centering
        \includegraphics[width=#1\columnwidth]{#2}
        \caption{#3}
        \label{#4}
    \end{figure}
}

% --- 双栏内紧凑表格环境 ---
\newcommand{\inserttablecompact}[3]{%
    \begin{table}[!htb]
        \centering
        \footnotesize  % 使用小字体
        #1
        \caption{#2}
        \label{#3}
    \end{table}
}

% --- 跨栏大表格环境 ---
\newcommand{\inserttablewide}[3]{%
    \begin{table*}[!tb]
        \centering
        \small  % 使用小字体
        #1
        \caption{#2}
        \label{#3}
    \end{table*}
}

% --- 并排子图环境(双栏内) ---
\newcommand{\insertsubfigs}[9][0.45]{%
    \begin{figure}[!htb]
        \centering
        \begin{subfigure}{#1\columnwidth}
            \centering
            \includegraphics[width=\textwidth]{#2}
            \caption{#3}
            \label{#4}
        \end{subfigure}
        \hfill
        \begin{subfigure}{#1\columnwidth}
            \centering
            \includegraphics[width=\textwidth]{#5}
            \caption{#6}
            \label{#7}
        \end{subfigure}
        \caption{#8}
        \label{#9}
    \end{figure}
}

% ========== 浮动体排版优化 ==========

% --- 图表标题格式 ---
% 双栏布局中的标题格式调整
\captionsetup[figure]{
    font=footnotesize,
    labelfont=bf,
    textfont=rm,
    justification=centering,
    singlelinecheck=false,
    skip=6pt
}

\captionsetup[table]{
    font=footnotesize,
    labelfont=bf,
    textfont=rm,
    justification=centering,
    singlelinecheck=false,
    skip=6pt
}

% --- 跨栏图表标题格式 ---
\DeclareCaptionStyle{widefigure}{
    font=small,
    labelfont=bf,
    textfont=rm,
    justification=centering,
    singlelinecheck=false,
    skip=8pt
}

\captionsetup[figure*]{style=widefigure}
\captionsetup[table*]{style=widefigure}

% ========== 特殊浮动体处理 ==========

% --- 侧边注释框(仅限单栏内) ---
\newcommand{\sidenotebox}[1]{%
    \begin{figure}[!htb]
        \centering
        \fcolorbox{blue}{blue!10}{%
            \parbox{0.9\columnwidth}{%
                \footnotesize
                \textbf{提示:}#1
            }
        }
    \end{figure}
}

% --- 算法环境适配双栏 ---
\floatstyle{ruled}
\newfloat{algorithm}{!htb}{loa}
\floatname{algorithm}{算法}

% 双栏算法环境
\newcommand{\insertalgorithm}[3]{%
    \begin{algorithm}[!htb]
        \centering
        \footnotesize
        #1
        \caption{#2}
        \label{#3}
    \end{algorithm}
}

% 跨栏算法环境
\newfloat{algorithm*}{!tb}{loa}
\floatname{algorithm*}{算法}

\newcommand{\insertalgorithmwide}[3]{%
    \begin{algorithm*}[!tb]
        \centering
        \small
        #1
        \caption{#2}
        \label{#3}
    \end{algorithm*}
}

% ========== 浮动体调试命令 ==========
% 用于调试浮动体位置的命令
\newcommand{\showfloatstatus}{%
    \typeout{当前浮动体状态:}%
    \typeout{  topnumber: \the\c@topnumber}%
    \typeout{  bottomnumber: \the\c@bottomnumber}%
    \typeout{  totalnumber: \the\c@totalnumber}%
    \typeout{  topfraction: \topfraction}%
    \typeout{  bottomfraction: \bottomfraction}%
    \typeout{  textfraction: \textfraction}%
}    % 浮动体优化
% =============================================================================
% 栏平衡功能配置文件
% Column Balance Feature Configuration
% =============================================================================
% 文件用途:双栏布局中的栏平衡功能配置
% 创建日期:2025-08-03
% 版本:v1.0
% 
% 使用说明:
% 1. 本文件提供双栏布局中的栏平衡功能
% 2. 在主文档的 preamble 中使用 % =============================================================================
% 栏平衡功能配置文件
% Column Balance Feature Configuration
% =============================================================================
% 文件用途:双栏布局中的栏平衡功能配置
% 创建日期:2025-08-03
% 版本:v1.0
% 
% 使用说明:
% 1. 本文件提供双栏布局中的栏平衡功能
% 2. 在主文档的 preamble 中使用 % =============================================================================
% 栏平衡功能配置文件
% Column Balance Feature Configuration
% =============================================================================
% 文件用途:双栏布局中的栏平衡功能配置
% 创建日期:2025-08-03
% 版本:v1.0
% 
% 使用说明:
% 1. 本文件提供双栏布局中的栏平衡功能
% 2. 在主文档的 preamble 中使用 \input{preamble/layout_double/balance.tex} 引入
% 3. 建议在引入其他双栏配置文件之后引入本文件
% 4. 栏平衡可确保双栏页面左右栏高度尽可能相等
% 
% 配置内容包括:
% - 自动栏平衡设置
% - 手动栏平衡命令定义
% - 栏平衡的触发条件配置
% - 最后一页的栏平衡处理
% - 栏平衡与浮动体的协调
% - 栏平衡算法的参数调整
% 
% 常用命令(将在配置中定义):
% - \balancecolumns : 强制当前页面栏平衡
% - \nobalancecolumns : 取消自动栏平衡
% - \lastpagebalance : 最后一页栏平衡
% 
% 注意事项:
% - 栏平衡可能影响页面排版效果
% - 与某些浮动体设置可能产生冲突
% - 建议根据具体文档内容调整平衡策略
% - 在文档最终定稿前进行栏平衡调试
% =============================================================================

% 此文件将在后续迁移过程中填充具体的栏平衡配置内容 引入
% 3. 建议在引入其他双栏配置文件之后引入本文件
% 4. 栏平衡可确保双栏页面左右栏高度尽可能相等
% 
% 配置内容包括:
% - 自动栏平衡设置
% - 手动栏平衡命令定义
% - 栏平衡的触发条件配置
% - 最后一页的栏平衡处理
% - 栏平衡与浮动体的协调
% - 栏平衡算法的参数调整
% 
% 常用命令(将在配置中定义):
% - \balancecolumns : 强制当前页面栏平衡
% - \nobalancecolumns : 取消自动栏平衡
% - \lastpagebalance : 最后一页栏平衡
% 
% 注意事项:
% - 栏平衡可能影响页面排版效果
% - 与某些浮动体设置可能产生冲突
% - 建议根据具体文档内容调整平衡策略
% - 在文档最终定稿前进行栏平衡调试
% =============================================================================

% 此文件将在后续迁移过程中填充具体的栏平衡配置内容 引入
% 3. 建议在引入其他双栏配置文件之后引入本文件
% 4. 栏平衡可确保双栏页面左右栏高度尽可能相等
% 
% 配置内容包括:
% - 自动栏平衡设置
% - 手动栏平衡命令定义
% - 栏平衡的触发条件配置
% - 最后一页的栏平衡处理
% - 栏平衡与浮动体的协调
% - 栏平衡算法的参数调整
% 
% 常用命令(将在配置中定义):
% - \balancecolumns : 强制当前页面栏平衡
% - \nobalancecolumns : 取消自动栏平衡
% - \lastpagebalance : 最后一页栏平衡
% 
% 注意事项:
% - 栏平衡可能影响页面排版效果
% - 与某些浮动体设置可能产生冲突
% - 建议根据具体文档内容调整平衡策略
% - 在文档最终定稿前进行栏平衡调试
% =============================================================================

% 此文件将在后续迁移过程中填充具体的栏平衡配置内容   % 栏平衡优化

% --- 双栏专用命令 ---
% 跨栏注意框
\newcommand{\noteboxwide}[1]{% 
    \begin{figure*}[!b]
    \centering
    \fcolorbox{red}{yellow!20}{% 
        \parbox{0.9\textwidth}{% 
            \textbf{注意:}#1 
        } 
    } 
    \end{figure*}
}

% 双栏代码输入命令
\newcommand{\inputcodewide}[3][python]{%
    \begin{figure*}[!t]
        \inputminted{#1}{#2}
        \caption{#3}
    \end{figure*}
}

% ========== 页眉页脚设置 ==========
\lhead{\includegraphics[width=34.7mm]{figure/badge-horizonal.pdf}}
\rhead{\righthead}
\cfoot{\thepage}

% ========== 宽度变量定义 ==========
% 注意事项框宽度
\newlength{\noteboxwidth}
\setlength{\noteboxwidth}{0.9\columnwidth}

% 自适应表格宽度
\newlength{\tablewidth}
\setlength{\tablewidth}{\columnwidth}

% ========== 参考文献设置 ==========
\addbibresource{main.bib}