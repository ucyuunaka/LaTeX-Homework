% ========== 个人配置文件 ==========
% 本文件包含所有可自定义的个人信息和文档设置

% ========== 个人信息设置 ==========
\newcommand{\school}{您的学院}
\newcommand{\major}{您的年级专业}
\newcommand{\righthead}{页眉标题}
\newcommand{\maintitle}{文档标题}
\newcommand{\course}{课程名称}
\newcommand{\teacher}{指导教师}
\newcommand{\expnumber}{报告编号}
\newcommand{\name}{姓名1、姓名2、姓名3}
\newcommand{\id}{学号1、学号2、学号3}
\newcommand{\customdate}{2025年12月20日}

% ========== 模板类型选择 ==========
% 取消注释对应行来选择不同的模板类型
% \newcommand{\templatetype}{实验报告}
% \newcommand{\templatetype}{课程作业}
% \newcommand{\templatetype}{课程设计}
\newcommand{\templatetype}{学术论文}

% ========== 布局选择设置 ==========
% 取消注释对应行来选择不同的文档布局
% single: 单栏布局(适合传统报告、作业等)
% double: 双栏布局(适合学术论文、期刊文章等)
\newcommand{\documentlayout}{single}
% \newcommand{\documentlayout}{double}

% 布局选择说明:
% - single: 使用单栏布局,页边距较宽,适合阅读和批注
% - double: 使用双栏布局,版面紧凑,适合正式发表
% 注意:切换布局后可能需要调整图表大小和代码块格式

% ========== 其他自定义设置 ==========
% 如果需要修改某些格式参数,可以在这里覆盖默认设置
% 例如:
% \geometry{left=3cm,right=3cm}  % 修改页边距
% \doublespacing                  % 修改为双倍行距 